%\documentclass{beamer}
\documentclass[usenames,dvipsnames]{beamer}

% Setup appearance:
\usetheme{Boadilla}
\usefonttheme[onlylarge]{structurebold}
\setbeamerfont*{frametitle}{size=\normalsize,series=\bfseries}
\setbeamertemplate{navigation symbols}{}
\setbeamertemplate{footline}{}
%\usepackage[para]{footmisc}
\setbeamertemplate{blocks}[rounded][shadow=false]

\usepackage{mwe,tikz}\usepackage[percent]{overpic}
\usepackage{tikz}

% Standard packages
\usepackage[english]{babel}
%\usepackage[latin1]{inputenc}
\usepackage{times}
\usepackage[T1]{fontenc}
\usepackage{amsmath}
\usepackage{subfig}
\usepackage{multimedia}
%\usepackage{graphicx}
\usepackage{media9}
%\usepackage{animate}
\usepackage{centernot}
\usepackage{changepage}
\usepackage{xcolor}
\usepackage[export]{adjustbox}
\usepackage[bottom]{footmisc}
\usepackage{tcolorbox}
\newtcolorbox{mybox}[3][]
             {
               colframe = black!75,
               colback  = #2!10,
               coltitle = #2!20!white,
               title    = {#3},
               #1,
             }

\usepackage{enumitem}

\usepackage[absolute,overlay]{textpos}
  \setlength{\TPHorizModule}{1mm}
  \setlength{\TPVertModule}{1mm}

\newenvironment<>{varblock}[2][.9\textwidth]{
  \setlength{\textwidth}{#1}
  \begin{actionenv}#3
    \def\insertblocktitle{#2}
    \par
    \usebeamertemplate{block begin}}
  {\par
    \usebeamertemplate{block end}
  \end{actionenv}}

%\usepackage{cite}
%\let\cite={\scriptsize \color{blue} \cite}
\newcommand{\ccite}[1]{{\color{blue}\scriptsize \cite{#1}}}
\newcommand{\pcite}[1]{{\color{blue}\scriptsize [#1]}}
\newcommand{\gcite}[1]{{\color{lightgray}\scriptsize [#1]}}
\newcommand{\BP}{{\rm BP}}
\newcommand{\ES}{{\rm ES}}

\definecolor{light-gray}{gray}{0.8}

%\usepackage[colorlinks=true,citecolor=blue,urlcolor=blue]{hyperref}
%\definecolor{links}{HTML}{2A1B81}
%\hypersetup{colorlinks,linkcolor=,citecolor=blue}

\def\stBar{\text{ }|\text{ }}
\def\bfx{ {\bf x} }
\def\bfc{ {\bf c} }
\def\bfv{ {\bf v} }
\def\bfvt{ \tilde{\bf v} }
\def\bfu{ {\bf u} }
\def\bfn{ {\bf n} }
\def\bfs{ {\bf s} }
\def\bff{ {\bf f} }
\def\bfq{ {\bf q} }

\usepackage[framemethod=TikZ]{mdframed}
\mdfdefinestyle{MyFrame}{%
    linecolor=blue,
    outerlinewidth=1pt,
    roundcorner=5pt,
    innertopmargin=\baselineskip,
    innerbottommargin=\baselineskip,
    innerrightmargin=20pt,
    innerleftmargin=20pt,
    backgroundcolor=white!50!white}

\usepackage{empheq}
\newcommand*\widefbox[1]{\fbox{\hspace{0.75em}#1\hspace{0.75em}}}

\setbeamersize{text margin left=20pt,text margin right=20pt}
% Setup TikZ
\usepackage{tikz}
\usetikzlibrary{arrows}
\tikzstyle{block}=[draw opacity=0.7,line width=1.4cm]

% Title
\title{
  {\normalsize AMCS 394E: Contemp. Topics in Computational Science.}
    \\
      {\normalsize Computing with the finite element method}}
\author{\scriptsize David I. Ketcheson and Manuel Quezada de Luna}
\date{}

% The main document
\begin{document}

%\maketitle
\begin{frame}
  \maketitle
  \vspace{90pt}  
  %\begin{textblock}{40}(10,45)
  %  \begin{figure}
  %    \includegraphics[scale=0.065]{figures/malpasset_mesh1.png}
  %  \end{figure}
  %\end{textblock}
%
  %\begin{textblock}{40}(70,50)
  %  \begin{figure}
  %    \includegraphics[scale=0.12]{figures/malpasset_soln_n0033_t1200.png}
  %  \end{figure}
  %\end{textblock}
%
  \begin{textblock}{40}(80,45)
    \begin{figure}
    \includegraphics[scale=0.25]{figures/kaust.jpg}
    \end{figure}
  \end{textblock}
\end{frame}

%%%%%%%%%%%%
\begin{frame}{\underline{Overview of the course}}
  \begin{textblock}{120}(5,15)
    Our goal is to learn the basics of the finite element (FE) 

    method and its practical implementation. 
  \end{textblock}
\end{frame}

\begin{frame}{\underline{Overview of the course}}
  \begin{textblock}{120}(5,15)
    Our goal is to learn the basics of the finite element (FE) 

    method and its practical implementation. 
  \end{textblock}

  \begin{textblock}{120}(5,32.5)
    {\color{red} \bf Objectives}
    \begin{itemize}
    \item[*]<2-> Learn the basics of the theory behind the FE method.
    \item[*]<3-> Learn how to implement the FE method from scratch. 
    \item[*]<4-> Learn to use a FE library to solve real applications.
    \end{itemize}
  \end{textblock}
\end{frame}

\begin{frame}{\underline{Overview of the course}}
  \begin{textblock}{120}(5,15)
    Our goal is to learn the basics of the finite element (FE) 

    method and its practical implementation. 
  \end{textblock}

  \begin{textblock}{120}(5,32.5)
    {\color{red} \bf Objectives}
    \begin{itemize}
    \item[*] Learn the basics of the theory behind the FE method.
    \item[*] Learn how to implement the FE method from scratch. 
    \item[*] Learn to use a FE library to solve real applications.
    \end{itemize}
  \end{textblock}

  \begin{textblock}{120}(5,62.5)
    {\color{red} \bf Prerequisites}    
    \begin{itemize}
    \item[*]<2-> Theoretical side: have exposure to PDEs and numerical methods.
    \item[*]<3-> Computational side: have exposure to Python or Matlab and C++. 
    \end{itemize}
  \end{textblock}
\end{frame}
%%%%%%%%%%%%%

%%%%%%%%%%%%%
\begin{frame}{\underline{Overview of the course}: method of evaluation}
  \begin{textblock}{120}(5,15)
    {\color{red} \bf Homeworks} (40\%)
    \begin{itemize}
      \item[*] Programming assignments with a short report.        
    \end{itemize}
    \end{textblock}
\end{frame}

\begin{frame}{\underline{Overview of the course}: method of evaluation}
  \begin{textblock}{120}(5,15)
    {\color{red} \bf Homeworks} (40\%)
    \begin{itemize}
      \item[*] Programming assignments with a short report.        
    \end{itemize}
    \end{textblock}

  \begin{textblock}{120}(5,35)
  {\color{red} \bf Final project} (60\%)
  \begin{itemize}
  \item[*]<2-> Objective: 
    {\scriptsize solve a practical application.}
  \item[*]<3-> Examples: 
      {\scriptsize
      \begin{itemize}
        \item[-] Solve the Euler equations in multiple dimensions. 
        \item[-] Test different h-adaptivity criteria during the sol. of the shallow water equations. 
        \item[-] Solve a problem of a floating cube via ALE.
        \item[-] Implement and test a numerical scheme from some publication.
        \item[-] Solve a equation or test a method from your own research. 
      \end{itemize}
      }
    \item[*]<4-> Format: 
      {\scriptsize report (around 10 pages) with a description of the problem, 
        equations, numerical methods, details of the implementation and numerical results.}
  \end{itemize}
  \end{textblock}
\end{frame}
%%%%%%%%%%%%%


%%%%%%%%%%%%%
\begin{frame}{\underline{Review of PDEs}}
  \begin{textblock}{120}(5,15)
    A PDE is an equation involving multiple independent variables 
    and their derivatives; e.g., consider 
    \begin{align*}
      u=u(x,y,t), \qquad F(u,u_t,u_x,u_{xx}, \dots, u_{y}, u_{yy}, \dots,x,y,t)=0.
    \end{align*}
  \end{textblock}
\end{frame}

\begin{frame}{\underline{Review of PDEs}}
  \begin{textblock}{120}(5,15)
    A PDE is an equation involving multiple independent variables 
    and their derivatives; e.g., consider 
    \begin{align*}
      u=u(x,y,t), \qquad F(u,u_t,u_x,u_{xx}, \dots, u_{y}, u_{yy}, \dots,x,y,t)=0.
    \end{align*}
  \end{textblock}

  \begin{textblock}{110}(10,46)
    \begin{center}
    ``Since Newton, mankind has come to realize that the laws of physics are always 
    expressed in the language of differential equations.''
    \end{center}
  \end{textblock}
  \begin{textblock}{40}(85,60)
    Steven Strogatz
  \end{textblock}
\end{frame}

\begin{frame}{\underline{Review of PDEs}}
  \begin{textblock}{120}(5,15)
    A PDE is an equation involving multiple independent variables 
    and their derivatives; e.g., consider 
    \begin{align*}
      u=u(x,y,t), \qquad F(u,u_t,u_x,u_{xx}, \dots, u_{y}, u_{yy}, \dots,x,y,t)=0.
    \end{align*}
  \end{textblock}

  \begin{textblock}{110}(10,46)
    \begin{center}
    ``Since Newton, mankind has come to realize that the laws of physics are always 
    expressed in the language of differential equations.''
    \end{center}
  \end{textblock}
  \begin{textblock}{40}(85,60)
    Steven Strogatz
  \end{textblock}

  \begin{textblock}{120}(5,75)
    \begin{center}
      {\scriptsize
        (0:45-2:04) \url{https://www.youtube.com/watch?v=p_di4Zn4wz4}
        \\
        (0:11-1:36, 3:32-5:58) \url{https://www.youtube.com/watch?v=ly4S0oi3Yz8}
      }
    \end{center}
  \end{textblock}
\end{frame}
%%%%%%%%%%%%%

%%%%%%%%%%%%%
\begin{frame}{\underline{Review of PDEs}}
  \begin{textblock}{120}(5,15)
    A PDE is an equation involving multiple independent variables 
    and their derivatives; e.g., consider 
    \begin{align*}
      u=u(x,y,t), \qquad F(u,u_t,u_x,u_{xx}, \dots, u_{y}, u_{yy}, \dots, x,y,t)=0.
    \end{align*}
  \end{textblock}

  \begin{textblock}{120}(5,42.5)
    PDEs model the bahavior of functions of multiple variables. 
    The applications are vast:
  \end{textblock}
  
  \begin{textblock}{120}(5,53.5)
    {\scriptsize
    \begin{itemize}
    \item[*] Weather forecast
    \item[*] Electromagnetism
    \item[*] Combustion
    \item[*] Distribution of stress in a structure
    \item[*] Finance
    \item[*] Nuclear physics
    \item[*] Blood flow 
    \item[*] Elasticity
    \end{itemize}
    }
  \end{textblock}

  \begin{textblock}{120}(65,53.5)
    {\scriptsize
    \begin{itemize}
    \item[*] Fluid flows
    \item[*] Multiphase flows
    \item[*] Floating objects (ships, etc)
    \item[*] Propagation of a tsunami
    \item[*] Quantum mechanics
    \item[*] Spacetime and matter relation 
    \item[*] Heat distribution and evolution 
    \item[*] Etcetera
    \end{itemize}
    }
  \end{textblock}
\end{frame}
%%%%%%%%%%%%%
%%%%%%%%%%%%%%%%%%%%%%%%%%%%
% ***** END OF SLIDE ***** %
%%%%%%%%%%%%%%%%%%%%%%%%%%%%

% ********************************* %
% ********** START SLIDE ********** %
% ********************************* %
\begin{frame}{\underline{Review of PDEs}}
  \begin{textblock}{120}(5,15)
    {\bf \color{red} Some important concepts:} 
    \begin{itemize}
      \item[*]<2-> Order of a PDE. 
      \item[*]<3-> Linear vs nonlinear PDEs.
      \item[*]<4-> Constant or variable coefficient PDEs. 
    \end{itemize}
  \end{textblock}
\end{frame}

\begin{frame}{\underline{Review of PDEs}}
  \begin{textblock}{120}(5,15)
    {\bf \color{red} Some important concepts:} 
    \begin{itemize}
      \item[*] Order of a PDE. 
      \item[*] Linear vs nonlinear PDEs.
      \item[*] Constant or variable coefficient PDEs. 
    \end{itemize}
  \end{textblock}

  \begin{textblock}{120}(5,50)
    {\bf \color{red} Types of PDEs based on:}
    \begin{itemize}
    \item[*]<2-> Order: first-order, second-order, etc.
    \item[*]<3-> Depending on the linearity: linear, quasilinear or nonlinear. 
    \item[*]<4-> Depending on the force term: homogeneous or non-homogeneous. 
    \end{itemize}
  \end{textblock}
\end{frame}
%%%%%%%%%%%%%

%%%%%%%%%%%%%
\begin{frame}{\underline{Review of PDEs}}
  \begin{textblock}{120}(5,15)
    {\bf \color{red} Classification of second order PDEs:}
    \begin{itemize}
      \item[*]<2-> Hyperbolic
      \item[*]<3-> Elliptic
      \item[*]<4-> Parabolic
    \end{itemize}
  \end{textblock}
\end{frame}

%%%%%%%%%%%%%%%
\begin{frame}{\underline{Numerical methods for PDEs}}
  \begin{textblock}{120}(5,15)
    {\color{red} \bf Most common numerical methods:}
    \begin{itemize}
      \item[*]<2-> Finite differences
      \item[*]<3-> Finite volumes 
      \item[*]<4-> Finite elements
    \end{itemize}
  \end{textblock}
\end{frame}

\begin{frame}{\underline{Numerical methods for PDEs}}
  \begin{textblock}{120}(5,15)
    {\color{red} \bf Most common numerical methods:}
    \begin{itemize}
      \item[*] Finite differences
      \item[*] Finite volumes 
      \item[*] Finite elements
    \end{itemize}
  \end{textblock}

  \begin{textblock}{120}(5,50)
    {\bf \color{red} Method of lines: }

    \vspace{5pt}
    Popular strategy to solve time dependent PDEs. 
    The spatial derivatives are discretized directly which leads to 
    system of (non)linear ordinary differential equations. 

    \vspace{10pt}
    The system of ODEs is solved at different time intervals. 
  \end{textblock}
\end{frame}

%%%%%%%%%%%%
\begin{frame}{\underline{Finite difference method (FDM)}}
  \begin{textblock}{120}(5,15)
    The FDM solves for point values at given nodes by 
    
    approximating the derivatives through finite differences.
  \end{textblock}
\end{frame}

\begin{frame}{\underline{Finite difference method (FDM)}}
  \begin{textblock}{120}(5,15)
    The FDM solves for point values at given nodes by 
    
    approximating the derivatives through finite differences.
  \end{textblock}
  
  \begin{textblock}{120}(5,40)
    {\bf \color{red} Some characteristics are:}
    \begin{itemize}
    \item[*]<2-> derivation and error analysis are done via Taylor series
    \item[*]<3-> easy to implement for structured grids.
    \item[*]<4-> fast and efficient
    \item[*]<5-> cumbersome for unstructured grids 
  \end{itemize}
  \end{textblock}
\end{frame}

%%%%%%%%%%%%%%
\begin{frame}{\underline{Finite difference method (FDM)}}
  \begin{textblock}{120}(5,15)
    Approximation of first- and second-order derivatives. 
  \end{textblock}
\end{frame}

\begin{frame}{\underline{Finite difference method (FDM)}}
  \begin{textblock}{120}(5,15)
    Approximation of first- and second-order derivatives. 
  \end{textblock}

  \begin{textblock}{120}(5,35)
    Example. Consider the following advection reaction diffusion equation:
    \begin{align*}
      u_t + u_x - \mu u_{xx} = r(x), \qquad \mu>0
    \end{align*}
  \end{textblock}
\end{frame}

\begin{frame}{\underline{Finite difference method (FDM)}}
  \begin{textblock}{120}(5,15)
    Approximation of first- and second-order derivatives. 
  \end{textblock}

  \begin{textblock}{120}(5,35)
    Example. Consider the following advection reaction diffusion equation:
    \begin{align*}
      u_t + u_x - \mu u_{xx} = r(x), \qquad \mu>0
    \end{align*}

    \begin{textblock}{120}(5,55)
      {\color{red} \bf Basic steps:}
      \begin{itemize}
      \item[*]<2-> Consider the spatial discretization for a given node $i$. 
      \item[*]<3-> Consider a matrix form of the semi-discretization. 
      \item[*]<4-> Discretize in time using e.g. BE, FE, RK methods, etc. 
      \end{itemize}
    \end{textblock}    
  \end{textblock}
\end{frame}

%%%%%%%%%%%%%
\begin{frame}{\underline{Finite volume method (FVM)}}
  \begin{textblock}{120}(5,15)
    The FVM solves for cell averages. 

    The derivatives are 
    commonly approximated via numerical fluxes. 
  \end{textblock}
\end{frame}


\begin{frame}{\underline{Finite volume method (FVM)}}
  \begin{textblock}{120}(5,15)
    The FVM solves for cell averages. 

    The derivatives are 
    commonly approximated via numerical fluxes. 
  \end{textblock}
  
  \begin{textblock}{120}(5,40)
    {\bf \color{red} Some characteristics are:}
  \begin{itemize}
    \item[*]<2-> easy to implement for structured grids
    \item[*]<3-> still relatively fast and efficient
    \item[*]<4-> high-order approximations are cumbersome for unstructured grids 
    \item[*]<5-> widely used for hyperbolic equations
  \end{itemize}
  \end{textblock}
\end{frame}

%%%%%%%%%%%%%%%%%
\begin{frame}{\underline{Finite volume method (FVM)}}
  \begin{textblock}{120}(5,15)
    Example. Consider the following advection reaction diffusion equation:
    \begin{align*}
      u_t + u_x - \mu u_{xx} = r(x), \qquad \mu>0
    \end{align*}
  \end{textblock}
\end{frame}

%%%%%%%%%%%%%%%%%
\begin{frame}{\underline{When to use each type of method?}}
  \begin{textblock}{120}(5,15)
    {\scriptsize
      {\bf \color{red} Finite differences:}
    \begin{itemize}
    \item[*]<2-> Easiest to implement and the most efficient
      \item[*]<3-> Difficult to generalize for unstructured grids
      \item[*]<4-> Easy to consider different type of operators 
      \item[*]<5-> Good to test different models with simple geometries
    \end{itemize}
    }
  \end{textblock}
\end{frame}

\begin{frame}{\underline{When to use each type of method?}}
  \begin{textblock}{120}(5,15)
    {\scriptsize
      {\bf \color{red} Finite differences:}
    \begin{itemize}
      \item[*] Easiest to implement and the most efficient
      \item[*] Difficult to generalize for unstructured grids
      \item[*] Easy to consider different type of operators 
      \item[*] Good to test different models with simple geometries
    \end{itemize}
    }
  \end{textblock}

  \begin{textblock}{120}(5,40)
    {\scriptsize
      {\bf \color{red} Finite volumes:}
      \begin{itemize}
      \item[*]<2-> FVM is widely used and developed for hyperbolic equations
      \item[*]<3-> Easy to implement and more efficient than FEM  
      \item[*]<4-> Difficult to generalize to unstructured grids
      \item[*]<5-> In some codes it is difficult to consider diffusive terms
      \end{itemize}
    }
  \end{textblock}
\end{frame}

\begin{frame}{\underline{When to use each type of method?}}
  \begin{textblock}{120}(5,15)
    {\scriptsize
      {\bf \color{red} Finite differences:}
    \begin{itemize}
      \item[*] Easiest to implement and the most efficient
      \item[*] Difficult to generalize for unstructured grids
      \item[*] Easy to consider different type of operators 
      \item[*] Good to test different models with simple geometries
    \end{itemize}
    }
  \end{textblock}

  \begin{textblock}{120}(5,40)
    {\scriptsize
      {\bf \color{red} Finite volumes:}
      \begin{itemize}
      \item[*] FVM is widely used and developed for hyperbolic equations
      \item[*] Easy to implement and more efficient than FEM  
      \item[*] Difficult to generalize to unstructured grids
      \item[*] In some codes it is difficult to consider diffusive terms
      \end{itemize}
    }
  \end{textblock}

  \begin{textblock}{120}(5,65)
    {\scriptsize
      {\bf \color{red} Finite elements:}
      \begin{itemize}
        \item[*]<2-> FEM is the slowest and harder to understand and implement.  
        \item[*]<3-> Easy to work with unstructured grids. 
        \item[*]<4-> Easy to consider complex geometries. 
        \item[*]<5-> Easy to consider different type of operators. 
      \end{itemize}
    }
    \end{textblock}
\end{frame}

%%%%%%%%%%%%%%%%%
\begin{frame}{\underline{Scientific libraries for finite elements}}
  \begin{textblock}{120}(5,15)
  Finite element libraries versus personalized codes. 
  
  {\scriptsize See \url{https://www.math.colostate.edu/~bangerth/videos.676.1.html}}
  \end{textblock}
\end{frame}

\begin{frame}{\underline{Scientific libraries for finite elements}}
  \begin{textblock}{120}(5,15)
  Finite element libraries versus personalized codes. 
  
  {\scriptsize See \url{https://www.math.colostate.edu/~bangerth/videos.676.1.html}}
  \end{textblock}

  \begin{textblock}{120}(5,40)
    {\bf \color{red} Some things to consider:}
    {\scriptsize
    \begin{itemize}
    \item[*]<2-> A personalized solver might be considerably faster.
    \item[*]<3-> Coding a fast solver from scratch requires considerable experience and knowledge. 
    \item[*]<4-> Some FE libraries have been largely tested and many bugs have been found and fixed. 
    \item[*]<5-> Libraries have many (difficult to implement) tools like: 
      \begin{itemize}
      \item[-] multigrid methods
      \item[-] parallelized solvers
      \item[-] h-adaptivity
      \item[-] moving meshes
      \item[-] etc
      \end{itemize}
    \item[*]<6-> In general it might take years to code all the tools available in a good FE library. 
    \end{itemize}
    }
  \end{textblock}
\end{frame}


%%%%%%%%%%%%%%%
\begin{frame}{\underline{Scientific libraries for finite elements}}
  \begin{textblock}{120}(5,15)
    {\bf \color{red} Examples of finite element libraries:}

    {\color{white} dummy} deal.II, MFEM, FEniCS Project, FreeFEM, Proteus, etc.
  \end{textblock}
\end{frame}

\begin{frame}{\underline{Scientific libraries for finite elements}}
  \begin{textblock}{120}(5,15)
    {\bf \color{red} Examples of finite element libraries: }

    {\color{white} dummy} deal.II, MFEM, FEniCS Project, FreeFEM, Proteus, etc.
  \end{textblock}

  \begin{textblock}{120}(5,30)
    {\bf \color{red} Common things between libraries:}
    {\scriptsize
    \begin{itemize}
      \item[*]<2-> Finite element loop: elements, quad points, shape functions, etc. 
      \item[*]<3-> Finite element transformations. 
      \item[*]<4-> Maps from local to global entries in the algebraic systems. 
      \item[*]<5-> Tools to perform quadratures, access data, etc. 
      \item[*]<5-> etc.
    \end{itemize}
    }
  \end{textblock}
\end{frame}

\begin{frame}{\underline{Scientific libraries for finite elements}}
  \begin{textblock}{120}(5,15)
    {\color{red}\bf Examples of finite element libraries: }

    {\color{white} dummy} deal.II, MFEM, FEniCS Project, FreeFEM, Proteus, etc.
  \end{textblock}

  \begin{textblock}{120}(5,30)
    {\color{red} \bf Common things between libraries:}
    {\scriptsize
    \begin{itemize}
      \item[*] Finite element loop: elements, quad points, shape functions, etc. 
      \item[*] Finite element transformations. 
      \item[*] Maps from local to global entries in the algebraic systems. 
      \item[*] Tools to perform quadratures, access data, etc. 
      \item[*] etc.
    \end{itemize}
    }
  \end{textblock}

  \begin{textblock}{120}(5,62.5)
    {\color{red} \bf Main differences:}
    {\scriptsize
    \begin{itemize}
      \item[*]<2-> Extra tools to mesh, solve the linear systems, visualization, etc. 
      \item[*]<3-> Pre-define functions to assemble common operators. 
      \item[*]<4-> How exposed are the indices, data structures, conectivity matrix, etc. 
      \item[*]<5-> Tools to handle matrices in different forms (CSR, standard row-column indices, etc). 
      \item[*]<5-> etc. 
    \end{itemize}
    }
  \end{textblock}
\end{frame}

%%%%%%%%%%%%%%%%
\begin{frame}{\underline{Scientific libraries for finite elements}}
  \begin{textblock}{120}(5,15)
    {\color{red} \bf What to look to choose a finite element library:}
    {\scriptsize
      \begin{itemize}
      \item[*]<2-> programming language 
      \item[*]<3-> software dependencies
      \item[*]<4-> parallelization
      \item[*]<5-> standard versus non-standard implementations
      \item[*]<5-> etc
      \end{itemize}
    }
  \end{textblock}
\end{frame}

\begin{frame}{\underline{Scientific libraries for finite elements}}
  \begin{textblock}{120}(5,15)
    {\color{red} \bf What to look to choose a finite element library:}
    {\scriptsize
      \begin{itemize}
      \item[*] programming language 
      \item[*] software dependencies
      \item[*] parallelization
      \item[*] standard versus non-standard implementations
      \item[*] etc
      \end{itemize}
    }
  \end{textblock}

  \begin{textblock}{120}(5,50)
    {\bf \color{red} Other libraries to work with finite elements:}
  {\scriptsize
    \begin{itemize}
    \item[*]<2-> meshing
    \item[*]<3-> partition of the domain
    \item[*]<4-> linear algebra
    \item[*]<5-> visualization
    \item[*]<5-> etc
    \end{itemize}
  }
  \end{textblock}
\end{frame}


%%%%%%%%%%%%%%%%%%%%%%%%%%%%
% ***** END OF SLIDE ***** %
%%%%%%%%%%%%%%%%%%%%%%%%%%%%



\end{document}


